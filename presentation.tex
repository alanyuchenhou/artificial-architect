\documentclass[12pt]{article}
\renewcommand*{\familydefault}{\sfdefault}
\usepackage{listings}
\usepackage[usenames,dvipsnames,svgnames,table]{xcolor}
%% \usepackage{color}
\usepackage{amsmath}
\usepackage{amsthm}
\usepackage{fullpage}
\usepackage{tabularx}
\usepackage{graphicx}
\usepackage{caption}
%% \usepackage{subfig}
\usepackage{subfigure}
%% \usepackage{cite}
\theoremstyle{definition}
\newtheorem{exmp}{Example}[section]
\begin{document}
\lstset{
  breaklines=true,
  commentstyle=\color{red},
  stringstyle=\color{orange},
  identifierstyle=\color{blue}
  keywordstyle=\color{violet},
  frame=single,
  language=Python}
\title{Architect project}
\author{Machine learning enhanced on-chip network design automation}
%% \author{Yuchen Hou}
\maketitle
\begin{figure}[h]
  \centering
      {\includegraphics[width=0.5\textwidth]{chips.jpg}}
\end{figure}
\pagebreak

\section{What's the problem?}
\pagebreak

\subsection{Electronics and chips}
\subsubsection{Chips are the hearts of electronic devices}
\subsubsection{Chips carry integrated circuits}
\begin{figure}[h]
  \centering
  \begin{subfigure}
    {\includegraphics[width=0.4\textwidth]{electronics.jpg}}
  \end{subfigure}
  \begin{subfigure}
    {\includegraphics[width=0.4\textwidth]{chip.jpg}}
  \end{subfigure}
\end{figure}
\pagebreak

\subsection{Chip design}
\subsubsection{Old paradigm: \\ gazillions of components = registers+amplifiers+capacitors+resisters+...}
\subsubsection{New paradigm: \\ computation+communication = cores+network}
\begin{figure}[h]
  \centering
  \begin{subfigure}
    {\includegraphics[width=0.4\textwidth]{circuits.png}}
  \end{subfigure}
  \begin{subfigure}
    {\includegraphics[width=0.4\textwidth]{noc.png}}
  \end{subfigure}
\end{figure}
\pagebreak

\subsection{On-chip networks}
\subsubsection{Goal: deliver packages from A to B}
\subsubsection{Metrics: latency, throughput/bandwidth, energy efficiency...}
\begin{figure}[h]
  \centering
  \begin{subfigure}
    {\includegraphics[width=0.5\textwidth]{mesh}}
  \end{subfigure}
\end{figure}
\pagebreak

\subsection{The problem : On-chip network design automation}
\pagebreak

\section{Why is this problem important? \\ -Beacuse it is useful, and hard!}
\subsection{Challenge: tradeoff between many conflicting features}
\subsection{Challenge: no analytical solution}
\subsection{Challenge: expensive simulation}
\subsection{Challenge: design complexity explosion}
\subsection{Challenge: limited designer productivity}
\pagebreak

\section{How to solve this problem?}
\pagebreak

\subsection{Existing approach: an iterative process}
\subsubsection{The designer analyzes the current design, and consider a few similar new designs}
\subsubsection{The designer assesses the qualities of new designs using his knowledge}
\subsubsection{The designer chooses a design with high quality and test it with a simulator}
\subsubsection{The designer examines the simulation report and improves his knowledge}
\subsubsection{If this design is good, then he's done! Otherwise, he repeats the process.}
\begin{figure}[h]
  \centering
  \begin{subfigure}
    {\includegraphics[width=0.9\textwidth]{topologies}}
  \end{subfigure}
\end{figure}
\pagebreak

\subsection{Questions to think about}
\subsubsection{What kind of problem is he solving? \\ -Optimization problem.}
\subsubsection{What algorithm is he using? \\ -Local search.}
\subsubsection{What makes designs better and better? \\ -Knowledge improvement.}
\subsubsection{Can machines do this job? \\ -Probably. That means we should...}
\pagebreak

\subsection{Consider using machine learning}
\begin{description}
  \item[designer] learner
  \item[design] data instance
  \item[knowledge] hypothesis
\end{description}
\pagebreak

\subsection{Existing approach: an iterative process}
\subsubsection{The designer analyzes the current design, and consider a few similar new designs}
\subsubsection{The designer assesses the qualities of new designs using his knowledge}
\subsubsection{The designer chooses a design with high quality and test it with a simulator}
\subsubsection{The designer examines the simulation report and improves his knowledge}
\subsubsection{If this design is good, then he's done! Otherwise, he repeats the process.}
\pagebreak

\subsection{New approach: still an iterative process}
\subsubsection{The learner analyzes the current data, and consider a few similar new data}
\subsubsection{The learner assesses the qualities of new data instances using its hypothesis}
\subsubsection{The learner chooses a data instance with high quality and test it with a simulator}
\subsubsection{The learner examines the simulation report and improves its hypothesis}
\subsubsection{If this data instance is good, then it's done! Otherwise, it repeats the process.}
\pagebreak

\section{A concrete example}
\subsection{Network designs}
\begin{figure}[h]
  \centering
  \begin{subfigure}
    {\includegraphics[width=0.4\textwidth]{graph2.png}}
  \end{subfigure}
  \begin{subfigure}
    {\includegraphics[width=0.4\textwidth]{graph3.png}}
  \end{subfigure}
\end{figure}
\begin{figure}[h]
  \centering
  \begin{subfigure}
    {\includegraphics[width=0.4\textwidth]{graph4.png}}
  \end{subfigure}
  \begin{subfigure}
    {\includegraphics[width=0.4\textwidth]{graph5.png}}
  \end{subfigure}
\end{figure}
\pagebreak

\subsection{Training data}
[data] = [quality] [features]\\
quality = energy\\
features = 1:degree 2:average-path-length 3:diameter 4:radius\\
\\
2.86818e-05 1:6 2:2.45337301587 3:4 4:3\\
2.71615e-05 1:6 2:2.46974206349 3:4 4:3\\
2.86003e-05 1:6 2:2.48859126984 3:4 4:3\\
2.76535e-05 1:6 2:2.48015873016 3:4 4:3\\
2.73489e-05 1:5 2:2.70138888889 3:4 4:4\\
2.86277e-05 1:4 2:3.29216269841 3:6 4:4\\
\pagebreak

\subsection{Model learning(SVM) \\ design.quality = hypothesis(design.features)}
\pagebreak

\subsection{Local search: hill climbing search}
\begin{figure}[h]
  \centering
  \begin{subfigure}
    {\includegraphics[width=\textwidth]{search}}
  \end{subfigure}
\end{figure}
\pagebreak

\section{What can you learn from my work?}
\subsection{Machine learning is everywhere, even in on-chip network design}
\subsection{Local search is the fundamental for optimization problems}
\subsection{Understanding the problem is important}
\subsection{New methods are not far away from existing methods}

\end{document}
